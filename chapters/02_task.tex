% !TeX root = ../main.tex
% Add the above to each chapter to make compiling the PDF easier in some editors.

\chapter{Task}\label{chapter:task}
The foundation and main focus of this work will be a thorough security analysis of the GSMA eSIM specification. Questions like
\begin{itemize}
    \item What are the requirements of eSIM for the individual systems
    \item How does the eSIM and the server mutually authenticate each other
    \item How does a profile push/pull work, what information is required and how is it authenticated
    \item What communication does exist and how are these (secure) channels built
    \item Are common attacks (e.g. Man in the middle, Dolev-Yao) properly averted
    \item Can keys and certificates be compromised and are there appropriate measures to handle this
\end{itemize}
and more must be answered. It is necessary to investigate in detail whether there are weaknesses in the specification and how these can be eliminated.

The second step will be to evolve the current specification. The work should analyze, which components can be reused, which need improvements and which need a fundamental restructuring with a focus on the remote profile provisioning. The confidentiality and integrity of the MNO profile and usage data must be guaranteed at all times, which requires a mutual authentication of both the profile server and the hardware component (\acrshort{eUICC}). Relevant scientific backgrounds must be used to support the decisions.

The GSMA specification can also be questioned, as to why two standards with entirely different solutions and software-stacks for similar functionalities must co-exist.
A unification of both systems would greatly increase the ease of use and flexibility of the eSIM itself.
Therefore, the third step of this work will be to unify the specification for the two-model eSIM and try to merge M2M with Consumer such that one common framework supports both methods for profile provisioning. Such a model must consider the following aspects:
\begin{itemize}
    \item Simultaneously support a profile "push" (Server initiated) and "pull" (Device initiated)
    \item Key-management (if required) is simplified
    \item Specification is easy to implement and runs on target (resource constrained) devices
    \item All security concerns as defined in the spec-evolution of step two are satisfied
\end{itemize}
Finally, this system should then be implemented as a proof-of-concept. A current idea is to modify the consumer software stack to fit the requirements of the proposed model. 

